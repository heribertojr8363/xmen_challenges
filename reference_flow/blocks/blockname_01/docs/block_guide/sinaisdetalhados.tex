\newpage

\section{Descrição dos sinais da interface} % (fold)
\label{sec:descrição_detalhada_de_sinais}

\subsection{Interface de Controle}
\label{sec:if_controle}

\begin{table}[H]
  \centering
  \renewcommand\arraystretch{1.25}
  \caption{Sinais de interface de controle detalhados.}
  \vspace{2mm}
  \begin{tabular}{|P{4.0cm}|P{0.7cm}|P{1.5cm}|P{8cm}|}
    \hline
    \textbf{Sinal}        & \textbf{I/O} & \textbf{Tamanho} & \textbf{Descrição}                                    \\ \hline
    clk\_i                 & I            & 1                & Clock síncrono                                       \\ \hline
    rst\_ni                & I            & 1                & Reset assíncrono baixo ativo                         \\ \hline
    test\_en\_i            & I            & 1                & Chaveia todos os clocks para test                    \\ \hline
 \end{tabular} 
\end{table}
Para uma descrição mais detalhada dos sinais e informação sobre os sinais não citados, olhar a documentação da ARM, que pode ser encontrada na sessão de Bibliografia
\subsection{RISC-V}
\label{sec:riscv}
\begin{table}[H]
  \centering
  \renewcommand\arraystretch{1.25}
  \caption{Sinais de interface de controle do RISCV.}
  \vspace{2mm}
  \begin{tabular}{|P{4.0cm}|P{0.7cm}|P{1.5cm}|P{8cm}|}
    \hline
    \textbf{Sinal}        & \textbf{I/O} & \textbf{Tamanho} & \textbf{Descrição}                                    \\ \hline
    clk\_i                 & I            & 1                & Clock síncrono                                       \\ \hline
    rst\_ni                & I            & 1                & Reset assíncrono baixo ativo                          \\ \hline
     fetch\_ enable\_i               & I            & 1    & Permite que o core realize fetch da memória de instruções \\ \hline
     boot\_ addr\_i               & I            & 1    & Endereço no qual o boot é iniciado \\ \hline
      debug\_ halt\_i              & I            & 1                & Para o core para permitir o acesso a seus registradores internos                          \\ \hline
       debug\_ resume\_i                & I            & 1                & Retorna a funcionalidade do core, após este ter sido "haltado" \\ \hline
\end{tabular} 
\end{table}
\subsection{AXI}
\label{sec:axi}
\begin{table}[H]
  \centering
  \renewcommand\arraystretch{1.25}
  \caption{Sinais de interface de controle do RISCV.}
  \vspace{2mm}
  \begin{tabular}{|P{4.0cm}|P{0.7cm}|P{1.5cm}|P{8cm}|}
    \hline
    \textbf{Sinal}        & \textbf{I/O} & \textbf{Tamanho} & \textbf{Descrição}                                    \\ \hline
    clk\_i                 & I            & 1                & Clock síncrono                                       \\ \hline
    rst\_ni                & I            & 1                & Reset assíncrono baixo ativo                          \\ \hline
    slave               & I            & 1    & interface na qual são conectados os escravos(slaves) do sistema \\ \hline
     master              & I            & 1                & interface na qual são conectados os mestres(masters) do sistema \\ \hline
       test\_ en\_i                & I            & 1                & sinal de teste \\ \hline
\end{tabular} 
\end{table}

\subsection{JTAG}
\label{sec:jtag}
\begin{table}[H]
  \centering
  \renewcommand\arraystretch{1.25}
  \caption{Sinais de interface de controle do RISCV.}
  \vspace{2mm}
  \begin{tabular}{|P{4.0cm}|P{0.7cm}|P{1.5cm}|P{8cm}|}
    \hline
    \textbf{Sinal}        & \textbf{I/O} & \textbf{Tamanho} & \textbf{Descrição}                                    \\ \hline
    clk\_i                 & I            & 1                & Clock síncrono                                       \\ \hline
    rstn\_i                & I            & 1                & Reset assíncrono baixo ativo                          \\ \hline
    tclk\_i                 & I            & 1                & clock de teste, geralmente é mais lento que o clk\_i \\ \hline
    trstn\_i                & I            & 1                & Test reset \\ \hline
    tms               & I            & 1    & seleciona o modo de teste \\ \hline
       tdi              & I            & 1    & entrada serial do jtag \\ \hline
     tdo              & I            & 1    & saída serial do jtag \\ \hline

\end{tabular} 
\end{table}
% section descrição_detalhada_de_sinais (end)
