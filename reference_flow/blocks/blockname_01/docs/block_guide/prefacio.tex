\section{Prefácio} % (fold)
\label{sec:prefácio}
 Este documento apresenta informações sobre o projeto XCGen e seus blocos, utilizados no laboratório X-Men.
 Este documento não tem como objetivo dispensar a leitura e entendimento das especificações que são fornecidas pelos desenvolvedores.

\subsection{Convenções}

\begin{table}[H]
  \centering
  \renewcommand\arraystretch{1.25}
  \caption{Lista de convenções.}
  \vspace{0.3cm}
  \begin{tabular}{|P{2.5cm}|P{10.5cm}|}
    \hline
    \textbf{Convenção} & \textbf{Descrição} \\ \hline
    I                  & Quando aparece na coluna \textbf{I/O} em tabelas de descrição de sinais, significa que o sinal é uma entrada do módulo. \\ \hline
    O                  & Quando aparece na coluna \textbf{I/O} em tabelas de descrição de sinais, significa que o sinal é uma saída do módulo. \\ \hline
  \end{tabular}
\end{table}
\subsection{Contatos para correções ou dúvidas}
\begin{itemize}
\item marcio.almeida@embedded.ufcg.edu.br
\end{itemize}
\subsection{Bibliografia} % (fold)

Para maiores informações sobre qualquer um dos blocos, os links podem ser encontrados nos documentos a seguir:
\begin{itemize}
\item \url{http://www.gstitt.ece.ufl.edu/courses/fall15/eel4720_5721/labs/refs/AXI4_specification.pdf}
\item \url{https://riscv.org/technical/specifications/}
\item \url{https://en.wikipedia.org/wiki/JTAG}
\end{itemize}
\label{sub:bibliografia}

% subsection bibliografia (end)
\newpage
\subsection{Acrônimos e abreviações} % (fold)
% ######## init table ########
\begin{table}[h]
  \centering
  % distancia entre a linha e o texto
      {\renewcommand\arraystretch{1.25}
        \caption{Acrônimos e abreviações.}
        \vspace{0.3cm}
        \begin{tabular}{ l l }
          \cline{1-1}\cline{2-2}  
          \multicolumn{1}{|p{3.850cm}|}{\textbf{Acrônimo} \centering } &
          \multicolumn{1}{p{8cm}|}{\textbf{Significado} \centering }
          \\  
          \cline{1-1}\cline{2-2}  
          \multicolumn{1}{|p{3.850cm}|}{AXI4 \centering } &
          \multicolumn{1}{p{8cm}|}{{\it Advanced Extensible Interface} (
Interface Extensível Avançada) \centering }
         \\  
          \cline{1-1}\cline{2-2}  
          \multicolumn{1}{|p{3.850cm}|}{JTAG \centering } &
          \multicolumn{1}{p{8cm}|}{{\it Joint Test Action Group} (Grupo de ação de teste conjunto
) \centering }
\\
         \cline{1-1}\cline{2-2}  
          \multicolumn{1}{|p{3.850cm}|}{RISC \centering } &
          \multicolumn{1}{p{8cm}|}{{\it Reduced Instruction Set Computer} (Computador com um conjunto reduzido de instruções
) \centering }
\\
         \cline{1-1}\cline{2-2}  
          \multicolumn{1}{|p{3.850cm}|}{XCGen \centering } &
          \multicolumn{1}{p{8cm}|}{{\it Xmen Campina Grande Generator} (Gerador de SoC Desenvolvido em Campina Grande, no laboratório Xmen
) \centering }
     
          \\  
          \hline
          
      \end{tabular} }
\end{table}



\label{sub:acrônimos_e_abreviações}


%\vspace{4cm}
% subsection acrônimos_e_abreviações (end)




% subsection glossário (end)
